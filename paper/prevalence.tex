\section{Prevalence of DNS Censorship}\label{sec:prevalence}

We use our collected dataset described in Section~\ref{sec:methodology} to
understand the prevalence of DNS censorship around the world, in both IPv4
and IPv6.

%Many censors block using DNS injection~\cite{satellite,global}.


%\SomeCountryTable
\TabBaseRate
%\TabBaseRateMedian % Nah don't use this one;

% DNS censorship not the only kind.
% ...

Table~\ref{tab:base-rate} shows the top 27~countries that have 50 or more
resolver pairs (IPv4 and IPv6 interfaces) that geolocate within the country.
For each of our 7,000 resolver pair, we measure the base rate of censored domains
(from our list of 700+ domains). Table~\ref{tab:base-rate} breaks this down into
four categories:
\texttt{A} records to IPv4 resolvers,
\texttt{AAAA} to IPv4 resolvers,
\texttt{A} records to IPv6 resolvers,
and \texttt{AAAA} records to IPv6 resolvers.



Because our dataset is balanced (all the domains have both A and AAAA records,
and each resolver pair consists of an IPv4 and IPv6 address), if censorship is
applied uniformly, we would expect the rate of censorship to be uniform within a
country across records and resolver interfaces. However, we find several cases
where this is not true.

\paragraph{Thailand} One prominent example is Thailand, where 8.2\% of
\texttt{A} records were blocked by IPv4 resolvers, compared to around 1\% for
\texttt{AAAA} records or IPv6 resolvers in the country. The reason for this is
that many instances of IPv4 resolvers predominantly (or exclusively) saw
blocking on only \texttt{A} records, while their IPv6 pair saw little or no
blocking on either \texttt{A} or \texttt{AAAA} records. For instance, one of the
larger resolvers by blocked domains had an IPv4 endpoint that blocked
240 \texttt{A} records, but only 15 \texttt{AAAA} records from our domain list.
Meanwhile, its IPv6 counterpart did not block any domains (either \texttt{A} or
\texttt{AAAA} records).

% jq -r 'select(.resolver_country=="TH" and .id=="4934-A").blocked_domains | keys | .[]' ../../data/resolver-blocks-jan-aaaa.json | grep -e '-AAAA$' | wc -l
% Removed 3 domains manually: www.stratcom.mil, www.hud.gov, portal.hud.gov




\paragraph{China}
China shows an unusual preference \textbf{toward} blocking \texttt{AAAA}
records over \texttt{A} records. While it is known that the Great Firewall
can block \texttt{AAAA} records and injects IPv6 traffic, it is not clear why it
would block those records more than \texttt{A} records. Manual investigation reveals
21 domains that almost exclusively have their \texttt{AAAA} record blocked, but
not their \texttt{A} record. For example, \texttt{gmail.com}'s \texttt{AAAA}
record is blocked by over 95\% of resolvers in China, but the corresponding
\texttt{A} record for \texttt{gmail.com} is only blocked by 1\% of resolvers.
The other 20 domains have similar patterns, leading to China's slight preference
in blocking \texttt{AAAA} records over \texttt{A} records. We do not find any
instances of domains in China that are similarly exclusively blocked by
\texttt{A} record but not \texttt{AAAA}.

%{"domain":"www.gmail.com-AAAA","country_code":"CN","v4_censored_count":186,"v6_censored_count":185}
%{"domain":"www.gmail.com-A","country_code":"CN","v4_censored_count":4,"v6_censored_count":1}

%{"domain":"www.apple.com-AAAA","country_code":"CN","v4_censored_count":189,"v6_censored_count":191}
%{"domain":"www.apple.com-A","country_code":"CN","v4_censored_count":5,"v6_censored_count":8}

% ~/v6/v4vsv6/cmd/baseRate$ python3 q6.py < ./CN.json 
%blocked in AAAA but not A: 21 domains
%translate.google.com
%learndigital.withgoogle.com
%www.snapchat.com
%news.google.com
%play.google.com
%video.google.com
%picasa.google.com
%www.projectbaseline.com
%googleblog.com
%hangouts.google.com
%messages.android.com
%www.apple.com
%android.com
%google.dz
%raseef22
%qalam.withgoogle.com
%www.gmail.com
%doubleclick.net
%gmail.com
%www.hacktivismo.com
%allo.google.com
%blocked in A but not AAAA: 0 domains





\paragraph{Iran}
While Iran supports both IPv6 and IPv4, it is more effective at blocking IPv4
resolvers (both \texttt{A} and \texttt{AAAA} records). We find this is due to
many of the IPv6 resolvers in Iran are actually \textbf{6to4 bridges}: these
are IPv6 addresses that are not native IPv6, but instead an encoding of an IPv4
address. For example, sending an IPv6 packet to \texttt{2002:0102:0304::} will
send the packet encapsulated in an IPv4 packet to 1.2.3.4 (at the 6to4 gateway),
whose 32-bit address is encoded in the IPv6 address (\texttt{0102:0304}).
However, this can result in the packet passing encapsulated past the censor, who
will observe an IPv4 packet with a protocol field denoting IPv6 encapsulation,
instead of the usual UDP-carrying DNS. A naive censor may ignore such packets,
allowing the domain lookup through unchecked uncensored. We observe 272 IPv6
resolvers are actually 6to4 bridges in Iran, all of which block at lower
(but non-zero) rates, compared to their corresponding IPv4 resolver.
%The reason
%that these hosts still see blocking at all is that they occasionally must
%recursively resolve the host, which may induce censorship on the outbound
%packet. % (how did they get a correct cached entry in the first place tho?)

% $ cat ../../data/resolver-blocks-jan-aaaa.json | jq '[.resovler_ip, .resolver_country] | @tsv' -r | awk '{print $1, $2}' | grep '2002:' | awk '{print $2}' | sort | uniq -c | sort -rn | head
%    628 KR
%    467 US
%    364 FR
%    272 IR
%    255 RU
%    244 DE
%    225 VN
%    175 IN
%    146 CA
%    127 GB
%



% 5.56.132.126 2002:538:847e::538:847e

\paragraph{Trends}
We observe two general trends that apply to many censoring countries in our data.
First, \textbf{IPv4 resolvers are more heavily censored than IPv6 resolvers}.
This may be because censors in those regions only inspect IPv4 packets, or that
censors are more widely deployed on IPv4 networks. Second, \textbf{Native record
types are more heavily censored than non-native records}. In other words, an
\texttt{A} record requested from an IPv4 resolver is more likely to be censored
than a \texttt{AAAA} record from the same interface. But conversely, we find
that \texttt{AAAA} records are also more often censored on IPv6 resolvers than
\texttt{A} records. This could be because a censor only supports detecting
\texttt{AAAA} records in IPv6 traffic, and only \texttt{A} records in IPv4.


\if0
Notably, China, Russia, Iran, and Hong Kong can clearly be seen to host
resolvers that consistently block many domains, demonstrating these countries have
deployed largely uniform national censorship policies. On the other hand,
countries that are known to censor may still appear to have no uniformly blocked
domains. This could be due to a country primarily using a different technology
to censor (such as HTTP, SNI, or IP blocking), bias in the set of domains we
queried, or in the set of resolvers used.

One illustrative example is India, a country where widespread Internet censorship has
been observed and studied~\cite{singh2020india,Yadav2018a}. Given the number
of resolvers we discover and the country's proclivity to censor, we would expect to see
some domains censored across the country, but instead our data suggests no
domains are censored uniformly. One explanation for this discrepancy is that
India's censorship is not as centralized as in other countries: private ISPs in
India are given lists of URLs to block, but it is up to the ISP how to carry out
this blocking~\cite{Gosain2017a}, leading to heterogeneity in what is blocked,
what techniques are used, and how often updates occur.  Figure~\ref{fig:india}
shows a clustering of Indian resolvers by what domains they block, illustrating the non-uniformity in
conducting censorship in the country. We draw resolvers as nodes sized proportional
to their blocklist size, and draw an edge between two resolver
nodes if the set of domains they block has a high similarity (Levenshtein edit
distance less than a threshold). The largest censored list found in India contained 31
censored domains, while many other resolvers censored few or no domains.
% jq 'select(.resolver_country=="IN" and .id=="4335-A").blocked_domains | keys | .[]' ../../data/resolver-blocks-jan-aaaa.json -r | awk -F'-' '{print $1}' | sort | uniq | wc -l


\FigIndiaCluster
\fi


