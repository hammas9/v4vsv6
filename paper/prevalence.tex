\section{Prevalence of DNS Censorship}\label{sec:prevalence}


Before investigating the difference in censorship between IPv4 and IPv6, we
frist study \textbf{what countries censor DNS} in the first place?
We use our collected dataset described in Section~\ref{sec:methodology} to
answer this question.


%Many censors block using DNS injection~\cite{satellite,global}.


%\SomeCountryTable
\TabBaseRate

% DNS censorship not the only kind.
% ...

Table~\ref{tab:country-resolvers} shows the 27~countries that have more than 50
resolver pairs (IPv4 and IPv6 interfaces) that geolocate within the country.
For each of our 7,000 resolver pair, we send our list of 700+ domains (for both A
and AAAA records). We label a domain as \emph{censored} if over half of the
resolvers in a given country return a result consistent with blocking. This
helps to eliminate seemingly spurious censorship events, either due to geo-location
errors, configuration issues, or custom blocklists deployed at single resolvers.

Notably, China, Russia, Iran, and Hong Kong can clearly be seen to host
resolvers that consistently block many domains, demonstrating these countries have
deployed largely uniform national censorship policies. On the other hand,
countries that are known to censor may still appear to have no uniformly blocked
domains. This could be due to a country primarily using a different technology
to censor (such as HTTP, SNI, or IP blocking), bias in the set of domains we
queried, or in the set of resolvers used.

One illustrative example is India, a country where widespread Internet censorship has
been observed and studied~\cite{singh2020india,Yadav2018a}. Given the number
of resolvers we discover and the country's proclivity to censor, we would expect to see
some domains censored across the country, but instead our data suggests no
domains are censored uniformly. One explanation for this discrepancy is that
India's censorship is not as centralized as in other countries: private ISPs in
India are given lists of URLs to block, but it is up to the ISP how to carry out
this blocking~\cite{Gosain2017a}, leading to heterogeneity in what is blocked,
what techniques are used, and how often updates occur.  Figure~\ref{fig:india}
shows a clustering of Indian resolvers by what domains they block, illustrating the non-uniformity in
conducting censorship in the country. We draw resolvers as nodes sized proportional
to their blocklist size, and draw an edge between two resolver
nodes if the set of domains they block has a high similarity (Levenshtein edit
distance less than a threshold). The largest censored list found in India contained 31
censored domains, while many other resolvers censored few or no domains.
% jq 'select(.resolver_country=="IN" and .id=="4335-A").blocked_domains | keys | .[]' ../../data/resolver-blocks-jan-aaaa.json -r | awk -F'-' '{print $1}' | sort | uniq | wc -l


\FigIndiaCluster


