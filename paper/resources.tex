\section{Censorship of IPv4 and IPv6 Resource Records} \label{sec:resources}

\para{Overview.}
In this section, we focus on {\it identifying and characterizing differences in
the handling of IPv4 and IPv6 resource records} in DNS censorship deployments.
Specifically, we seek to answer the following questions: 
%
(\Cref{sec:resources:country}) In which countries is the censorship of IPv4
resource records (DNS {\tt A} queries) significantly different than the
censorship of IPv6 resource records (DNS {\tt AAAA} queries)?,
%
(\Cref{sec:resources:resolvers}) what are the characteristics of the resolvers
which exhibit differences in the handling of {\tt A} and {\tt AAAA} queries?,
and 
%
(\Cref{sec:resources:domains}) what are the characteristics of domains in which
these differences are frequently observed?
%

\subsection{Within-country differences in the censorship of
\texttt{A} and \texttt{AAAA} resource records} \label{sec:resources:country}
%
We use the responses received from our {\tt A} and {\tt AAAA} queries sent to
the same set of resolvers and for the same set of domains (\cf
\Cref{sec:methods:gathering} for a detailed description of our data gathering
process). We then apply the censorship determination methods described in
\Cref{sec:methods:labeling} to measure the prevalence of censorship on our {\tt
A} and {\tt AAAA} DNS queries. Finally, we perform statistical tests to
identify significant differences in the prevalence of censorship of {\tt A} and
{\tt AAAA} queries within each country.

\para{Identifying differences within a country.} 
To measure differences in DNS query handling within a specific country, we
compare the prevalence of censorship on {\tt A} and {\tt AAAA} queries by
aggregating responses across {each resolver within the country}. This presents
us with a two distributions (one each for the group of {\tt A} and {\tt AAAA}
queries) of the fraction of censored domains observed from resolvers in the
country.
%
We use a $t$-test to verify statistical significance of any observed
differences between the two groups for each country. In our statistical
analysis, we aim to achieve a significance level of 5\% ($p \leq$  .05)
\emph{over all our findings}. Therefore, we apply a Sidak correction
\cite{abdi2007bonferroni} to control for Type I errors from multiple hypothesis
testing. 
%
This requires $p \leq 1-{.05}^{1/n_{c}}$ for classifying a difference as
significant, where $n_c$ is the total number of countries in our dataset
(\fixme{XXX}). This approach reduces the likelihood of false-positive reports
of within-country differences.
%
The presence of a statistically significant difference for a specific country
would imply that the two resource types appear to undergo different censorship
mechanisms within that country (if a centralized mechanism for censorship
exists) or that a significant number of resolvers within that country have
inconsistencies in their censoring of each query type.
%
A summary of our results are presented in \Cref{tab:resources:countries}. 

\para{How many countries demonstrate large-scale inconsistencies in their
handling of {\tt A} and {\tt AAAA} queries?} 
%
In total, only seven countries showed a statistically significant difference
in the rate at which {\tt AAAA} and {\tt A} DNS requests were blocked. 
%
This finding suggests the presence of independent censorship mechanisms for
handling each query type in the seven countries (Thailand, Bangladesh,
Pakistan, Chile, Vietnam, Korea, and China).
%
Of these, six (China being the only exception) were found to have lower
blocking rates for {\tt AAAA} queries than {\tt A} queries. In fact, the {\tt
AAAA} censorship rates were between 36-78\% lower than the {\tt A} censorship
rate suggesting that their censorship mechanisms for {\tt AAAA} queries that
are associated with IPv6 connectivity are still lagging. 
%
Further analysis shows that the differences are mostly found on the IPv4
interfaces of our resolvers (\cf {\it IPv4 resolvers} column in
\Cref{tab:resources:countries}) where the {\tt AAAA} censorship rates were upto
86\% lower than the {\tt A} censorship rates. 
%
This finding is indicative of a tendency for network operators to have
focused efforts on maintaining infrastructure for censoring {\tt A} queries
sent to IPv4 resolvers, while paying less attention to the handling of {\tt
AAAA} queries and their IPv6 interfaces. It also presents an opportunity for
the developers of circumvention tools capable of dual-stack operations.
%
Once again, China presents the only exception with a preference for blocking
{\tt AAAA} queries on both, IPv4 and IPv6, interfaces of resolvers with a 10\%
and 13\% higher {\tt AAAA} censorship rate, respectively.

\begin{table}[t]
  \centering
  \small
  \scalebox{\tabularscale} {
  \begin{tabular}{lccc}%p{.9in}p{.9in}}
    \toprule
    {\bf Country}&{\bf IPv4 resolvers}&{\bf IPv6 resolvers} & {\bf All resolvers}
    \\ \midrule
    Thailand (TH)      & -7.1 pp (-85.7\%) & $ns$               & -3.7 pp (-77.5\%) \\
    Bangladesh (BD)    & -5.1 pp (-80.0\%) & $ns$               & -2.6 pp (-71.3\%) \\
    Pakistan (PK)      & -2.1 pp (-73.6\%) & -2.8 pp (-59.8\%)  & -2.5 pp (-60.2\%) \\
    Chile (CL)         & -2.0 pp (-57.6\%) & $ns$               & -1.1 pp (-58.9\%) \\
    Vietnam (VN)       & $ns$              & -0.7 pp (-52.5\%)  & -0.7 pp (-47.1\%) \\
    Korea (KR)         & -1.3 pp (-54.6\%) & $ns$               & -0.6 pp (-36.2\%) \\
    China (CN)         &  3.1 pp (+10.4\%) &  3.7 pp (+12.9\%)  &  3.4 pp (+11.7\%) \\
    \midrule
    United States (US) & -0.5 pp (-33.2\%) &  0.4 pp (+72.3\%)  &  $ns$  \\
    Myanmar (MY)       & -2.9 pp (-58.9\%) & $ns$    &  $ns$  \\
    \bottomrule
  \end{tabular}
  }
  \caption{Differences in blocking rates of {\tt A} and {\tt AAAA} queries
  observed over IPv4, IPv6, and all resolvers in a country. `pp' denotes the
  change in terms of percentage points (computed as {\tt AAAA} blocking rate
  - {\tt A} blocking rate) and the \%age value denotes the percentage change in
  blocking rate (computed as
  $
  100\times\frac{\text{{\tt AAAA} blocking rate} - \text{{\tt A} blocking rate}}
  {\text{{\tt A} blocking rate}}
  $). 
  Only countries having
  a statistically significant difference are reported. A negative value
  indicates that {\tt A} queries observed higher blocking rates than {\tt AAAA}
  queries. $ns$ indicates the difference was not statistically significant and
  thus omitted.}
  \label{tab:resources:countries}
\end{table}

\subsection{Characteristics of {\tt A/AAAA}-inconsistent resolvers}
\label{sec:resources:resolvers}
Given our above results which suggest that there are a number of countries in
which {\tt A} and {\tt AAAA} queries are censored differently, we now seek to
understand the characteristics of the resolvers that cause these differences.
%
We first focus on identifying the fraction of resolvers in each country that
have statistically different behaviors for {\tt A} and {\tt AAAA} queries.
Then, we compare the AS distributions of these resolvers with the set of all
resolvers in a country. This comparison allows us to hypothesize about the
presence of a centralized DNS censorship mechanism is responsible for the
observed inconsistencies or if they should be attributed to local network
operators responsible for implementing DNS censorship. Finally, we identify the
types of networks hosting inconsistent resolvers to get a measure of whether
users in residential networks may exploit these DNS inconsistencies for
circumvention.

\para{Identifying differences in individual resolvers.} 
%
We begin our analysis by identifying the individual resolver pairs (\ie we
consider the IPv4- and IPv6-interfaces of a resolver as a unit), within each
of the seven countries listed above, that have a statistically significant
difference in their censorship of {\tt A} and {\tt AAAA} queries.
%
To measure differences in DNS query handling of individual resolvers, we
compare the ratios of censored responses observed from our {\tt A} and {\tt
AAAA} queries {from each resolver}.  
%
We use a two-proportion $z$-test to verify the statistical significance of any
observed difference in the ratios between the two groups for each resolver.
Similar to our within-country analysis, we apply a Sidak correction to account
for our testing of multiple hypotheses and use $p \leq 1-{.05}^{1/n_{r_c}}$ to
classify a difference as significant, where $n_{r_c}$ is the total number of
resolvers in our dataset belonging to country $c$.
%
A summary of our results is provided in \Cref{tab:resources:resolvers}. 

\para{Which countries have the largest fractions of resolvers
exhibiting {\tt A} and {\tt AAAA} resolution inconsistencies?}
%
Immediately standing out from the other countries are Thailand, Bangladesh, and
Pakistan. These countries have {\tt A} and {\tt AAAA} inconsistencies in
between 62-82\% of their open resolvers. In comparison, other countries with
statistically significant differences have inconsistencies arising from
anywhere between 3-30\% of their resolvers. 
%

\para{How spread out are the {\tt A/AAAA}-inconsistent resolvers?}
We calculate the entropy of the distribution of all resolvers in the
country ($S_{\text{query}}^{\text{all}}$) and compare it with the entropy of
the distribution of the inconsistent resolvers in the country
($S_{\text{query}}^{\text{inconsistent}}$). This serves as a measure of the
diversity of ASes observed in both cases. 
%
In order to compare the two measures, we use the Kullback-Leibler divergence
($\nabla_{\text{query}}$) metric \cite{KLdivergence}. In simple terms, the
KL-divergence metric between two distributions ($X$, $Y$) measures the number
of additional bits required to encode $Y$ given the optimal encoding for $X$.
%
This computation is helpful for hypothesizing the censorship infrastructure
that causes the inconsistencies. Finding a small $\nabla_{\text{query}}$ value
in a country signifies that the inconsistent resolvers had a similar
distribution to all the resolvers in that country. This would suggest the
presence of a centralized mechanism that (roughly) equally impacts all ASes in
the country. 
Conversely, a higher $\nabla_{\text{query}}$ value indicates that there is
a strong change in the distribution of resolvers -- \ie a disproportionate
number of inconsistencies arise from a smaller set of ASes. This would be
indicative of local configuration inconsistency (at the network or resolver
level), rather than a centralized configuration inconsistency.
%
Based on this analysis, we once again see that Thailand, Bangladesh, and
Pakistan stand out with small $\nabla_{\text{query}}$ values (0.14 - 0.48).
This suggests the presence of a centralized censorship mechanism whose
inconsistency in handling of {\tt A} and {\tt AAAA} queries nearly equally
impacts all ASes. 
%
Korea, China, and the United States on the other hand demonstrate high
$\nabla_{\text{query}}$ scores suggesting the presence of network- or
resolver-level misconfigurations. This is confirmed by inspecting the ASes
hosting the resolvers with inconsistencies. For example, in the United States,
resolvers in just 5 ASes (of 249 ASes with resolvers) account for 56\% of all
{\tt A} and {\tt AAAA} inconsistencies.


\begin{table*}[t]
  \centering
  \small
  \scalebox{\tabularscale} {
    \begin{tabular}{lcclcccl}
    \toprule
      {\bf Country} & {\bf Total pairs} & {\bf Inconsistent pairs} & {\bf Most inconsistent AS} & \multicolumn{3}{c}{\bf AS diversity} & {\bf Most inconsistent type} \\ 
      & & {(\% of total pairs)}& {(\# inconsistent pairs)} & $S^{\text{all}}_{\text{query}}$ & $S^{\text{inconsistent}}_{\text{query}}$ & $\nabla_{\text{query}}$  & {(\# inconsistent pairs)} \\
      \midrule
      Thailand (TH)       & 186 & 152 (81.7\%)  &  AS9835 Government IT Services (40)  & 4.50 & 4.06 & 0.14 & Cable/DSL (110) \\
      Bangladesh (BD)     & 29  & 18 (62.1\%) & AS 9230 Bangladesh Online (4)          & 4.10 & 3.61 & 0.48 & Cable/DSL (18) \\    
      Pakistan (PK)       & 23  & 15 (65.2\%) & AS 17911 Brain Telecom (3)             & 3.43 & 3.06 & 0.25 & Cable/DSL (12) \\    
      Chile (CL)          & 65  & 20 (30.1\%) & AS 27651 Entel Chile (13)              & 3.08 & 1.14 & 1.18 & Corporate (13) \\    
      Vietnam (VN)        & 252 & 64 (25.4\%) & AS 131353 NhanHoa Software (37)        & 3.89 & 2.22 & 0.71 & Cable/DSL (59) \\    
      Korea (KR)          & 632 & 80 (12.7\%) & AS 9848 Sejong Telecom (13)            & 3.12 & 4.17 & 1.30 & Cable/DSL (58) \\    
      China (CN)          & 194 & 6 (3.1\%)   & AS 55933 Cloudie Limited (2)           & 3.89 & 2.25 & 2.56 & Corporate (4)   \\    
    \midrule
      United States (US)  & 1,228 & 175 (14.3\%)  & AS 30475 WEHOSTWEBSITES (35) & 6.28 & 4.69 & 1.31 & Corporate (129) \\    
      Myanmar (MY)        & 50  & 30 (60\%)   & AS 136170 Exabytes Network (10)  & 3.31 & 2.42 & 0.48 & Corporate (26) \\    
    \bottomrule
  \end{tabular}
  }
  \caption{Characteristics of the resolvers which demonstrated a statistically
  significant difference in their handling of {\tt A} and {\tt AAAA} queries in
  each country. 
  %
  `AS diversity' denotes the entropies of (all) resolver distribution
  ($S^{\text{all}}_{\text{query}}$) and {\tt A/AAAA}-inconsistent resolver
  distribution ($S^{\text{inconsistent}}_{\text{query}}$) across a country's
  ASes, and `$\nabla_{\text{query}}$' represents the Kullback-Leibler
  divergence of the distribution of inconsistent resolvers from the
  distribution of all resolvers in the country's ASes (\cf
  \Cref{sec:resources:resolvers}).
  %
  `Most inconsistent type' denotes the connection type with the most number of
  {\tt A/AAAA}-inconsistent resolvers.}
  \label{tab:resources:resolvers}
\end{table*}

\para{What types of networks exhibit the most {\tt A} and {\tt AAAA}
inconsistencies?}
%
We use the Maxmind GeoIP2 connection type database (retrieved in 01/2022
\cite{maxmind-connectiondb}) to identify the connection type of the resolvers
responsible for {\tt A} and {\tt AAAA} inconsistencies. 
%
We find that, in most countries, Cable/DSL network connections (typically
associated with residential networks) were most likely to host a resolver
exhibiting an inconsistency. 
%
Of the seven countries with statistically significant overall differences, only
Thailand and China were found to have a high ratio of {\tt A/AAAA} -inconsistent
resolvers in corporate networks. 
%
Combined with our previous results which suggest the presence of an
inconsistency in a centralized mechanism in Thailand, Bangladesh, and Pakistan,
these results show that these inconsistencies are likely extending to
residential networks --- a promising sign for the citizen users of
circumvention tools which exploit the {\tt A/AAAA} gap.

\subsection{Characteristics of anomalous domains} 
\label{sec:resources:domains}

\para{Identifying differences in domain behaviors within a country.} 
To uncover the domains which have their {\tt A} and {\tt AAAA} queries treated
differently within a country, we compare the ratios of censored responses in
each group for each domain across all resolvers within a country.  
%
We use a two-proportion $z$-test to verify the statistical significance of any
observed differences between the two query types for each domain tested within
each country. We once again apply a Sidak correction to ensure that our results
for domains identified to behave differently within each country are within
a 5\% significance level. We set $p \leq 1-{.05}^{1/n_{dom_{c}}}$, where
$n_{dom_{c}}$ represents the number of domains tested in country $c$.
%
In the absence of significant within-country differences, a statistically
significant difference for a domain would suggest the presence of
a domain-specific characteristic that results in the two resource records
undergoing different censorship mechanisms within the corresponding country.

\parait{Results.}

