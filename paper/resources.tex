\section{Censorship of IPv4 and IPv6 Resource Records} \label{sec:resources}

\para{Overview.}
In this section, we focus on {\it identifying and characterizing differences in
the handling of IPv4 and IPv6 resource records} in DNS censorship deployments.
Specifically, we seek to answer the following questions: 
%
(\Cref{sec:resources:country}) In which countries is the censorship of IPv4
resource records (DNS {\tt A} queries) significantly different than the
censorship of IPv6 resource records (DNS {\tt AAAA} queries)?,
%
(\Cref{sec:resources:resolvers}) what are the characteristics of the resolvers
which exhibit differences in the handling of {\tt A} and {\tt AAAA} queries?,
and 
%
(\Cref{sec:resources:domains}) what are the characteristics of domains in which
these differences are frequently observed?
%

\subsection{Within-country differences in the censorship of
\texttt{A} and \texttt{AAAA} resource records} \label{sec:resources:country}
%
We use the responses received from our {\tt A} and {\tt AAAA} queries sent to
the same set of resolvers and for the same set of domains (\cf
\Cref{sec:methods:gathering} for a detailed description of our data gathering
process). We then apply the censorship determination methods described in
\Cref{sec:methods:labeling} to measure the prevalence of censorship on our {\tt
A} and {\tt AAAA} DNS queries. Finally, we perform statistical tests to
identify significant differences in the prevalence of censorship of {\tt A} and
{\tt AAAA} queries within each country.

\para{Identifying differences within a country.} 
To measure differences in DNS query handling within a specific country, we
compare the prevalence of censorship on {\tt A} and {\tt AAAA} queries by
aggregating responses across {each resolver within the country}. This presents
us with a two distributions (one each for the group of {\tt A} and {\tt AAAA}
queries) of the fraction of censored domains observed from resolvers in the
country.
%
We use a $t$-test to verify statistical significance of any observed
differences between the two groups for each country. In our statistical
analysis, we aim to achieve a significance level of 5\% ($p \leq$  .05)
\emph{over all our findings}. Therefore, we apply a Sidak correction
\cite{abdi2007bonferroni} to control for Type I errors from multiple hypothesis
testing. 
%
This requires $p \leq 1-{.05}^{1/n_{c}}$ for classifying a difference as
significant, where $n_c$ is the total number of countries in our dataset
(\fixme{XXX}). This approach reduces the likelihood of false-positive reports
of within-country differences.
%
The presence of a statistically significant difference for a specific country
would imply that the two resource types appear to undergo different censorship
mechanisms within that country (if a centralized mechanism for censorship
exists) or that a significant number of resolvers within that country have
inconsistencies in their censoring of each query type.

\para{Results.} 
A summary of our results are presented in \Cref{tab:resources:countries}. In
total, only seven countries showed a statistically significant difference in
the rate at which {\tt AAAA} and {\tt A} DNS requests were blocked. 
%
This finding suggests the presence of independent censorship mechanisms for
handling each query type in the seven countries (Thailand, Bangladesh,
Pakistan, Chile, Vietnam, Korea, and China).
%
Of these, six (China being the only exception) were found to have lower
blocking rates for {\tt AAAA} queries than {\tt A} queries. In fact, the {\tt
AAAA} censorship rates were between 36-78\% lower than the {\tt A} censorship
rate suggesting that their censorship mechanisms for {\tt AAAA} queries that
are associated with IPv6 connectivity are still lagging. 
%
Further analysis shows that the differences are mostly found on the IPv4
interfaces of our resolvers (\cf {\it IPv4 resolvers} column in
\Cref{tab:resources:countries}) where the {\tt AAAA} censorship rates were upto
86\% lower than the {\tt A} censorship rates. 
%
This finding is indicative of a tendency for network operators to have
focused efforts on maintaining infrastructure for censoring {\tt A} queries
sent to IPv4 resolvers, while paying less attention to the handling of {\tt
AAAA} queries and their IPv6 interfaces. It also presents an opportunity for
the developers of circumvention tools capable of dual-stack operations.
%
Once again, China presents the only exception with a preference for blocking
{\tt AAAA} queries on both, IPv4 and IPv6, interfaces of resolvers with a 10\%
and 13\% higher {\tt AAAA} censorship rate, respectively.

\begin{table}[t]
  \centering
  \small
  \scalebox{\tabularscale} {
  \begin{tabular}{lccc}%p{.9in}p{.9in}}
    \toprule
    {\bf Country}&{\bf IPv4 resolvers}&{\bf IPv6 resolvers} & {\bf All resolvers}
    \\ \midrule
    Thailand (TH)      & -7.1 pp (-85.7\%) & $ns$               & -3.7 pp (-77.5\%) \\
    Bangladesh (BD)    & -5.1 pp (-80.0\%) & $ns$               & -2.6 pp (-71.3\%) \\
    Pakistan (PK)      & -2.1 pp (-73.6\%) & -2.8 pp (-59.8\%)  & -2.5 pp (-60.2\%) \\
    Chile (CL)         & -2.0 pp (-57.6\%) & $ns$               & -1.1 pp (-58.9\%) \\
    Vietnam (VN)       & $ns$              & -0.7 pp (-52.5\%)  & -0.7 pp (-47.1\%) \\
    Korea (KR)         & -1.3 pp (-54.6\%) & $ns$               & -0.6 pp (-36.2\%) \\
    China (CN)         &  3.1 pp (+10.4\%) &  3.7 pp (+12.9\%)  &  3.4 pp (+11.7\%) \\
    \midrule
    United States (US) & -0.5 pp (-33.2\%) &  0.4 pp (+72.3\%)  &  $ns$  \\
    Myanmar (MY)       & -2.9 pp (-58.9\%) & $ns$    &  $ns$  \\
    \bottomrule
  \end{tabular}
  }
  \caption{Differences in blocking rates of {\tt A} and {\tt AAAA} queries
  observed over IPv4, IPv6, and all resolvers in a country. `pp' denotes the
  change in terms of percentage points (computed as {\tt AAAA} blocking rate
  - {\tt A} blocking rate) and the \%age value denotes the percentage change in
  blocking rate (computed as
  $
  100\times\frac{\text{{\tt AAAA} blocking rate} - \text{{\tt A} blocking rate}}
  {\text{{\tt A} blocking rate}}
  $). 
  Only countries having
  a statistically significant difference are reported. A negative value
  indicates that {\tt A} queries observed higher blocking rates than {\tt AAAA}
  queries. $ns$ indicates the difference was not statistically significant and
  thus omitted.}
  \label{tab:resources:countries}
\end{table}

\subsection{Characteristics of inconsistent resolvers}
\label{sec:resources:resolvers}
Given our above results which suggest that there are a number of countries in
which {\tt A} and {\tt AAAA} queries are censored differently, we now seek to
understand the characteristics of the resolvers that cause these differences
--- specifically, are they spread out across many Autonomous Systems (ASes)
within a country? Finding that these resolvers are concentrated within a few
ASes would indicate which network operators have inconsistent mechanisms for
censoring {\tt A} and {\tt AAAA} queries, while a broad spread across many ASes
within the country would suggest the presence of a centralized censorship
mechanism which contains inconsistencies in the handling of {\tt A} and {\tt
AAAA} queries. 


\para{Identifying differences in individual resolvers.} 
%
We begin our analysis by identifying the individual resolver pairs (\ie we
consider the IPv4- and IPv6-interfaces of a resolver as a unit), within each
of the seven countries listed above, that have a statistically significant
difference in their censorship of {\tt A} and {\tt AAAA} queries.
%
To measure differences in DNS query handling of individual resolvers, we
compare the ratios of censored responses observed from our {\tt A} and {\tt
AAAA} queries {from each resolver}.  
%
We use a two-proportion $z$-test to verify the statistical significance of any
observed difference in the ratios between the two groups for each resolver.
Similar to our within-country analysis, we apply a Sidak correction to account
for our testing of multiple hypotheses and use $p \leq 1-{.05}^{1/n_r}$ to
classify a difference as significant, where $n_{r_c}$ is the total number of
resolvers in our dataset belonging to country $c$.

\para{Results.}

\begin{table*}[t]
  \centering
  \small
  \scalebox{\tabularscale} {
    \begin{tabular}{lccp{2in}cp{1.5in}}
    \toprule
      {\bf Country} & 
      {\bf Total pairs} & {\bf Inconsistent pairs} & 
      {\bf Most inconsistent AS} & 
      {\bf AS diversity ($\Delta_{AS}$)} &
      {\bf Most inconsistent type} 
    \\ \midrule
      Thailand (TH)       & 186 & 152 (81.7\%)  &  AS9835 Government IT Services (40/41)  & 4.50 (0.14) & Corporate (38/43) \\
      Bangladesh (BD)     & 29  & 18 (62.1\%) & AS 9230 Bangladesh Online (4/4)           & 4.10 (0.48) & Cable/DSL (18/29) \\    
      Pakistan (PK)       & 23  & 15 (65.2\%) & AS 17911 Brain Telecom (3/3)              & 3.06 (0.25) & Cable/DSL (12/15) \\    
      Chile (CL)          & 65  & 20 (30.1\%) & AS 27651 Entel Chile (13/13)              & 1.14 (1.18) & Cable/DSL (7/12) \\    
      Vietnam (VN)        & 252 & 64 (25.4\%) & AS 131353 NhanHoa Software (37/63)        & 2.22 (0.71) & Cable/DSL (60/207) \\    
      Korea (KR)          & 632 & 80 (12.7\%) & AS 9848 Sejong Telecom (13/13) & 4.17 (1.30) & Cable/DSL (58/452) \\    
      China (CN)          & 194 & 6 (3.1\%)   & AS 55933 Cloudie Limited (2/2) & 2.25 (2.56) & Corporate (4/15)   \\    
    \midrule
      United States (US)  & 1,228 & 175 (14.3\%)  & AS 30475 WEHOSTWEBSITES (35/36) & 4.69 (1.31) & Corporate (129/757) \\    
      Myanmar (MY)        & 50  & 30 (60\%)   & AS 136170 Exabytes Network (10/10) & 2.42 (0.48) & Corporate (26/30) \\    
    \bottomrule
  \end{tabular}
  }
  \caption{Characteristics of the resolvers which demonstrated a statistically
  significant difference in their handling of {\tt A} and {\tt AAAA} queries in
  each country. `AS diversity' denotes the entropy of inconsistent resolver
  distribution across a country's ASes and `$\Delta_{AS}$' represents its
  Kullback-Leibler divergence from the distribution of all resolvers in the
  country's ASes. `Most inconsistent type' denotes the connection type which
  experienced the most inconsistencies in {\tt A} and {\tt AAAA} query
  handling.}
  \label{tab:resources:resolvers}
\end{table*}





\subsection{Characteristics of anomalous domains} 
\label{sec:resources:domains}

\para{Identifying differences in domain behaviors within a country.} 
To uncover the domains which have their {\tt A} and {\tt AAAA} queries treated
differently within a country, we compare the ratios of censored responses in
each group for each domain across all resolvers within a country.  
%
We use a two-proportion $z$-test to verify the statistical significance of any
observed differences between the two query types for each domain tested within
each country. We once again apply a Sidak correction to ensure that our results
for domains identified to behave differently within each country are within
a 5\% significance level. We set $p \leq 1-{.05}^{1/n_{dom_{c}}}$, where
$n_{dom_{c}}$ represents the number of domains tested in country $c$.
%
In the absence of significant within-country differences, a statistically
significant difference for a domain would suggest the presence of
a domain-specific characteristic that results in the two resource records
undergoing different censorship mechanisms within the corresponding country.

\parait{Results.}

