\section{Censorship by Resource Record Types} \label{sec:resources}

\para{Overview.}
In this section, we focus on {\it identifying and characterizing differences in
the handling of IPv4 and IPv6 resource records} in DNS censorship deployments.
Specifically, we seek to answer the following questions: 
%
(\Cref{sec:resources:sig}) is the censorship of IPv4 resource records (DNS
{\tt A} queries) significantly different than the censorship of IPv6 resource
records (DNS {\tt AAAA} queries)?,
%
(\Cref{sec:resources:domains}) what are the characteristics of domains in which
these differences are frequently observed?, and
%
(\Cref{sec:resources:resolvers}) what are the characteristics of the resolvers
which frequently exhibit such differences?

\subsection{Quantifying differences in the censorship of \texttt{A} and
\texttt{AAAA} resource records} \label{sec:resources:sig}
%
We use the responses received from our {\tt A} and {\tt AAAA} queries sent to
the same set of resolvers and for the same set of domains (\cf
\Cref{sec:methods:gathering} for a detailed description of our data gathering
process). We then apply the censorship determination methods described in
\Cref{sec:methods:labeling} to measure the prevalence of censorship on our {\tt
A} and {\tt AAAA} DNS queries. Finally, we perform statistical tests to
identify significant differences at the:

\begin{itemize}
  \item {\it national-level.} To measure national-level differences in DNS query
    handling, we compare the prevalence of censorship on {\tt A} and {\tt AAAA}
    queries by aggregating responses across \emph{each resolver within a single
    country}. This presents us with a two distributions (one each for the group
    of {\tt A} and {\tt AAAA} queries) of the fraction of censored domains
    observed from resolvers in the country.
    %
    We use a $t$-test ($p$ < .05) to verify statistical significance of any
    observed differences between the two groups for each country. The presence
    of a statistically significant difference for a specific country would
    imply that the two resource types appear to undergo different censorship
    mechanisms within that country (or, a significant number of resolvers
    within that country have inconsistencies in their handling of each query
    type).

  \item {\it resolver-level.} To measure resolver-level differences in DNS
    query handling, we compare the ratios of censored responses observed from
    our {\tt A} and {\tt AAAA} queries \emph{from each resolver used in our
    study}. 
    %
    We use a two-proportion $z$-test ($p$ < .05) to verify the statistical
    significance of any observed difference in the ratios between the two
    groups for each resolver. In the absence of significant national-level
    differences, the presence of a statistically significant difference for
    a resolver would imply that the two resource types appear to undergo
    different censorship mechanisms due to resolver misconfiguration.

  \item {\it domain-level.} In this analysis, we compare the prevalence of
    censorship on the {\tt A} and {\tt AAAA} query by aggregating responses
    \emph{for each (domain, country) pair in our study}.
    %
    We use a $t$-test ($p$ < .05) to verify the statistical significance of any
    observed differences between the two query types for each (domain, country)
    pair. A statistically significant difference for a (domain would suggest
    the presence of a domain-specific characteristic that results in the two
    resource records undergoing different censorship mechanisms within the
    corresponding country.
\end{itemize}


