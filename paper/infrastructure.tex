\section{Censorship of IPv4 and IPv6 DNS Traffic}
\label{sec:infrastructure}

\para{Overview.} In this section, we focus on {\it identifying and
characterizing the differences in handling DNS queries sent over IPv4 and IPv6
networks} in DNS censorship mechanisms. 
%
Specifically, we answer the following questions:
%
(\Cref{sec:infrastructure:country}) In which countries are the DNS censorship
mechanisms for IPv4 and IPv6 traffic significantly different?,
%
(\Cref{sec:infrastructure:resolvers}) what are the characteristics of resolvers
that exhibit differences in the censorship of IPv4 and IPv6 traffic?, and
%
(\Cref{sec:infrastructure:domains}) what are the characteristics of the domains
in which such differences are frequently exhibited?

\subsection{Within-country differences in the censorship of IPv4 and IPv6 DNS
queries} \label{sec:infrastructure:country}

We use the responses received from the IPv4 and IPv6 interfaces of each of the
dual-stack resolvers in our dataset for the same set of domains. We then apply
the censorship detection mechanism detailed in \Cref{sec:methods:labeling} to
measure the prevalence of DNS censorship of queries sent over IPv4 and IPv6.
Finally, we perform statistical tests to identify the countries that have
significant differences in their censorship of IPv4 and IPv6 DNS traffic.

\para{Identifying differences within a country.} 
To measure differences in the censorship of DNS queries sent over IPv4 and
IPv6, we compare the prevalence of censorship on each by aggregating their
responses across each resolver within a country. This presents us with two
distributions (corresponding to the IPv4 and IPv6 interfaces of resolvers) of
the fraction of censored domains from resolvers within the corresponding
country.
%
We use a $t$-test to verify statistical significance of any observed
differences between the two groups for each country. Similar to our approach in
\Cref{sec:resources}, we apply a Sidak correction to control for Type I errors
from multiple hypothesis testing. 
%
This requires $p \leq 1-{.05}^{1/n_{c}}$ for classifying a difference as
significant, where $n_c$ is the total number of countries in our dataset
(\fixme{XXX}). 
%
The presence of a statistically significant difference for a specific country
would imply that the country appears to have different censorship mechanisms
for IPv4 and IPv6 DNS traffic (if a centralized mechanism for censorship
exists) or that a significant number of resolvers within that country are not
consistent in their censorship of IPv6 and IPv4 traffic.

\para{Results.}
A summary of our results are presented in \Cref{tab:infrastructure:countries}.
In total, we find only five countries (Thailand, Iran, Bangladesh, Myanmar, and
the United States) with statistically significant differences in their handling
of DNS queries over IPv4 and IPv6 traffic --- suggesting the use of independent
censorship mechanisms for IPv4 and IPv6. 
%
Interestingly, all these countries appear to have gaps in their IPv6 censorship
apparatus --- \ie IPv4 rates of blocking are higher than IPv6 rates in all
countries with significant differences. These differences result in IPv6
queries experiencing between 12\% and 78\% less censorship than their IPv4
counterparts.
%
Once again, this suggests a tendency for network operators to more effectively
maintain IPv4 DNS censorship infrastructure than IPv6 infrastructure. These
gaps present opportunities for the success of circumvention tools with IPv6
capabilities.  
%
Further analysis shows that these differences primarily arise due to the fact
that DNS type {\tt A} queries are significantly more likely to be blocked over
IPv4 connections than over an IPv6 connection. However, this is not the case
for {\tt AAAA} queries, with the exception of Iran.
%
Taken together, these findings are particularly noteworthy for circumvention
efforts in Thailand, Myanmar, and Iran where IPv6 adoption rates are high
(between 15\% and 45\%) and dual-stack tools may be used for circumvention of
DNS censorship.
%

\begin{table}[t]
  \centering
  \small
  \scalebox{\tabularscale} {
  \begin{tabular}{lccc}%p{.9in}p{.9in}}
    \toprule
    {\bf Country}&{\bf {\tt A} queries }&{\bf {\tt AAAA} queries} & {\bf All queries}
    \\ \midrule
    Thailand (TH)      & -7.1 pp (-86.3\%) & $ns$              & -3.7 pp (-78.1\%) \\
    Iran (IR)          & -3.2 pp (-12.6\%) & -3.0 pp (-12.5\%) & -3.1 pp (-12.5\%) \\ 
    Bangladesh (BD)    & -5.5 pp (-86.1\%) & $ns$              & -3.0 pp (-77.9\%) \\
    Myanmar (MY)       & -3.6 pp (-74.2\%) & $ns$              & -2.1 pp (-62.3\%) \\
    United States (US) & -0.9 pp (-64.8\%) & $ns$              & -0.5 pp (-42.6\%) \\
    \midrule
    Korea (KR)         & -1.1pp (-46.5\%) & $ns$    & $ns$ \\
    Chile (CL)         & -1.6pp (-58.7\%)  & $ns$    & $ns$ \\
    \bottomrule
  \end{tabular}
  }
  \caption{Differences in blocking rates of DNS queries sent to IPv4 and IPv6
  interfaces of each resolver in a country. `pp' denotes the change in
  terms of percentage points (computed as blocking rate of IPv6 - blocking
  rate of IPv4) and the \%age value denotes the percentage change in blocking rate
  (computed as 
  $
  100 \times \frac{\text{IPv6 blocking rate} - \text{IPv4 blocking rate}}
  {\text{IPv4 blocking rate}}
  $). 
  Only countries having a statistically
  significant difference are reported. A negative value indicates that queries
  sent over IPv4 observed higher blocking rates than those sent over IPv6. $ns$
  indicates the difference was not statistically significant and thus omitted.}
  \label{tab:infrastructure:countries}
\end{table}

\subsection{Characterization of inconsistent resolvers}
\label{sec:infrastructure:resolvers}


\subsection{Characterization of anomalous domains}
\label{sec:infrastructure:domains}

