\section{Background}\label{sec:background}


Network based censorship is a common barrier to global internet access today.
State actors have the ability to incorporate large scale network monitoring
systems into national network infrastructure preventing access to information
through either active interference or threat of retribution. There
have been many studies on censorship techniques both globally~\cite{} and within
individual countries~\cite{}.

Internet infrastructure is evolving to accommodate global connectivity as well
-- IPv6 deployment provides routable addressing to an increasing portion of the
internet as IPv4 allocations become scarce. Effective censorship strategies
attempting to control or limit accessibility have to keep up.
Censorship strategies are capable of reactive change as demonstrated by
the impact that real world events like elections~\cite{}, regime change~\cite{},
and specifically crafted circumvention tools~\cite{beznazwy2020china}
have on blocking techniques and block-lists. Documenting contemporary censorship
strategies through the transition from IPv4 to IPv6 can help to understand the
the trajectory of network censorship efforts more broadly.


\subsection{DNS Censorship Measurement}

The Domain Name System (DNS) underpins the global internet by providing a
mapping from human readable hostnames to routable IP addresses making domain name
resolution the first step in almost all connection establishment flows. However,
the widely deployed DNS system is implemented as a plaintext protocol allowing
on-path eavesdroppers to inspect the hostnames as clients attempt to establish
connections and in some cases inject falsified responses to interfere.

The Chinese traffic inspection system, called the Great Firewall (GFW),
is documented injecting falsified DNS responses as early as 2002~\cite{global2002great}.
This censorship has been shown to be a packet injection from an on-path adversary
monitoring for hostname substrings in DNS queries~\cite{}. An on-path adversary
operates on a copy of traffic transiting a specific link or gateway in a network.
This contrasts with an in-path adversary which operates on traffic inline with
the capability to drop or modify packets on the fly. These traffic inspection
devices can be housed at or near border gateways~\cite{xu2011internet} or
distributed throughout regional ISPs~\cite{ramesh2020decentralized} within
censoring countries.

Common strategies for measuring censorship via DNS injection involve routing DNS queries for
block-listed hostnames across a censoring link in a controlled environment where
the returned resource records can independently validated. Two notable previous
studies Satellite and Iris~\cite{scott2016satellite,pearce2017global} identify
open DNS resolvers across the internet by scanning the entire IPv4 space on port~53.
Satellite probes reliably available open resolvers from distributed set of vantage
points and detects incorrect or inconsistent response information. Iris develops
a method for identifying active manipulation in contrast to misconfiguration by
leveraging metrics such as consistency and independent verifiability. Iris
compares records returned by open resolvers to records returned a set of trusted
resolvers matching IP addresses, content hash, TLS certificate, and more. Our work
extends these strategies to investigate the consistency of DNS injection
between IPv4 and IPv6.

\subsection{IPv6 \& DNS}
\label{subsec:v4vsv6}

The proportion of clients that support IPv6 is rapidly growing, especially in
developing areas with newly deployed network infrastructure.
According to the APNIC internet registry over a quarter of the users
on the internet now route their traffic using IPv6~\cite{Huston-APNIC2021}.
Similarly Google metrics indicate that over 50\% of users access services using
IPv6 in India, Saudi Arabia, Germany and several other countries~\cite{Google-IPv6}.

\textbf{A vs AAAA.} DNS \texttt{A} queries and records respectively request and provide
resources to resolve hostnames into IPv4 addresses. \texttt{AAAA} queries and
records resolve hostnames to IPv6 addresses.

\textbf{IPv4 vs IPv6 Queries.}The type of resource requested is
indicated in the \texttt{query\_type} portion of a DNS request and is not linked
to the IP version that the DNS query is sent over - i.e. both \texttt{A} and
\texttt{AAAA} can be resolved over IPv4 or IPv6.
Some hostnames resolve exclusively to IPv4 or IPv6 i.e. only provide \texttt{A}
or \texttt{AAAA} records respectively, as is the case for many legacy sites on
the internet that only support IPv4.

A theoretically comprehensive DNS censorship strategy using response injection
requires traffic monitoring infrastructure to analyze both IPv4 and IPv6
traffic and parse both \texttt{A} and \texttt{AAAA} queries accounting for hostnames
that may not implement resource records of one type or the other. Such a strategy
would impede connections that use plaintext DNS (as outlined in the
original specifications~\cite{RFC1035,RFC3596}) - it would not effect more secure DNS
protocols such as DoT, DoH, or censorship resistance strategies which are not
covered in this work.

%% Are there appliances that inject responses to AAAA requests upon seeing a
%% block-listed domain name where the actual site doesn't support IPv6 - and
%% should never provide AAAA records in response?