\section{Dataset and Methodology}\label{sec:methodology}

In order to measure the prevalence of censorship over IPv6 we have three main
steps, which we explain in more detail in the following sections:
\begin{packed_enumerate}
    \item Identify IPv6 Resolvers (\Cref{ssec:ipv6-resolvers})
    \item Identify Domains of interest (\Cref{ssec:censored-domains})
    \item Determining DNS censorship (\Cref{ssec:determining-censorship})
\end{packed_enumerate}


\subsection{Identifying IPv6 Resolvers}\label{ssec:ipv6-resolvers} 

In order to identify IPv6 resolvers we rely upon the work of Hendricks et
al.~\cite{hendriks2017potential}. First we scan the entire IPv4 address space
using \texttt{zmap}~\cite{Durumeric13zmap} using a DNS query on port 53 for a
control domain that we control and is uncensored. This provides a list of
possible DNS resolvers over the IPv4 address space. 

We then issue an encoded DNS query to each IPv4 address for an \texttt{A} Record
of an uncensored domain that we control which is present in only a single Name
Server that we also control that is hosted only on an IPv6 address. As IPv4
addresses cannot directly communicate with IPv6 addresses the IPv4 resolver must
ask its preferred IPv6 address to ask our Name Server for the address of the
domain. Our encoded DNS query includes (as a subdomain) the IPv4 address of the
resolver we made our query to. Thus, if our domain that is hosted only in our
Name Server is \texttt{v6.tlsfingerprint.io} and we issued our \texttt{A} Record
Request to Cloud Flare's public DNS resolver \texttt{1.1.1.1} our query would be
for \texttt{1-1-1-1.v6.tlsfingerprint.io}.

By capturing packets on the server hosting our Name Server we can see \texttt{A}
record requests that come from IPv6 addresses for domains with embedded IPv4
addresses. This allows us to associate IPv6 addresses we see in our packet
capture with IPv4 resolvers.

We immediately see that some IPv6 addresses are assocaited with multiple IPv4
resolvers, \ie resolvers hosted on different IPv4 addresses will contact our Name
Server over the same IPv6 address. In order to properly measure the differences
between IPv6 and IPv4 resolvers we filter our list to IPv6 addresses that were
only seen as a result of a single IPv4 resolver's query, \ie we ignore IPv6
addresses that are associated with multiple IPv4 resolvers.

Furthermore, the requests our Name Server receives are not guaranteed to come
from resolvers. Thus we perform a follow up scan by issuing \texttt{A} and
\texttt{AAAA} Record Requests to each IP address of each pair of IPv4 and IPv6
address for uncensored domains we control using
\texttt{zdns}~\cite{Durumeric13zmap}. By examining the responses we can further
filter our list to pairs of (IPv6, IPv4) resolvers that responded correctly to
DNS queries to uncensored control domains over both Records.

We then use Maxmind\fixme{cite} to geolocate the IPv4 and IPv6 resolvers by
country. And further filter our list to those (IPv6, IPv4) resolver pairs that
are located in the same country. Finally we filter our list of resolver pairs to
those in countries with at least three resolver pairs allowing us to have
multiple queries within a country on each Network Interface.

This reduction leaves us with 7,788 resolver pairs from 89 countries to be the
basis of our measurements.


\subsection{Identifying Domains of interest}\label{ssec:censored-domains}

Censored Planet's Satellite project~\cite{sundara2020censored} has cultivated a
list of domains that are potentially censored globally or by individual
countries. This list provides a starting point for identifying domains of
interest to our study.

To measure how censorship varies over IPv4 and IPv6 Network Interfaces we
restrict our list of domains to those that have both \texttt{A} and
\texttt{AAAA} Records. To filter this list we again use \texttt{zdns} to perform
both \texttt{A} and \texttt{AAAA} Record Requests for each domain from
Satellite's global list through Google and Cloud Flare's public DNS resolvers
(\texttt{8.8.8.8, 8.8.4.4, 1.1.1.1,} and \texttt{1.0.0.1}).

We filter our domain list to those domains that have valid IPv4 addresses in
their \texttt{A} Records and valid IPv6 addresses in their \texttt{AAAA}
Records.

To facilitate the next step of determining when a domain is censored we take the
Answers provided from our DNS queries and issue follow up TLS connections using
\texttt{zgrab2}~\cite{Durumeric13zmap} and locally verify the returned TLS
certificates.

Our domain list is then filtered to those domains that:
\begin{packed_enumerate}
    \item Have an IPv4 address in their \texttt{A} Record
    \item Have an IPv6 address in their \texttt{AAAA} Record
    \item Issue a valid TLS certificate for the domain on both the IPv4 and IPv6
    address in their \texttt{A/AAAA} Records
\end{packed_enumerate}

On January 24, 2022 this gave us 717 domains, originally identified by Censored
Planet as potentially interesting from a censorship point of view that meet our
criteria to measure censorship across network interfaces.

\subsection{Determining DNS censorship}\label{ssec:determining-censorship}

To measure DNS censorship across Network Interfaces we then perform both
\texttt{A} and \texttt{AAAA} Record Requests for our filtered 717 domains from
each IP address of all 7,788 resolver pairs.

We then take each answer provided by every resolver and make a TLS connection
with the Server Name of the domain we queried the resolver for and retrieve any
returned TLS certificate.

We then verify each DNS query we made to determine if any answers provided
failed to successfully return a valid TLS certificate, if so we label the DNS
query from that resolver for that domain and resource Record as censored.
